% VFD Short Symbolic Review (SSR)
% LaTeX Version for Overleaf

\documentclass[12pt]{article}
\usepackage{amsmath, amssymb}
\usepackage{geometry}
\usepackage{titlesec}
\usepackage{graphicx}
\usepackage{setspace}
\usepackage{hyperref}
\geometry{margin=1in}
\setstretch{1.2}

\title{\textbf{Vibrational Field Dynamics (VFD)} \\ \large Short Symbolic Review (SSR)\\ Version 0.2 -- Public Symbolic Overview}
\author{Lee Smart \ Contact: contact@vibrationalfielddynamics.org \ X: @vfd_org}
\date{}

\begin{document}
\maketitle

\begin{abstract}
Vibrational Field Dynamics (VFD) proposes that the foundation of physical, biological, and cognitive coherence arises from a unified geometric substrate characterized by $\varphi$-scaling, torsional intervals, and recursive standing-wave stability. Rather than viewing matter, space and information as separate categories, VFD frames them as stable geometric patterns in a deformable vibrational medium. This document provides a symbolic, non-disclosive introduction suitable for researchers exploring geometry-based models of synchrony, invariants, and cross-scale organisation. All generative mathematics, operators, and internal mechanisms are intentionally omitted.
\end{abstract}

\section{Motivation}
Increasing evidence across physics, neuroscience, and biology suggests the presence of deeper geometric invariants underlying coherence phenomena. Examples include:

\begin{itemize}
    \item Stable phase-locked neural waves that exhibit scale-free properties and $\varphi$-like ratios.
    \item Cross-brain synchrony and geometric invariants in interbrain networks.
    \item Molecular vibrational modes constrained by geometry-based interference conditions.
    \item Self-organising biological systems whose behaviour reflects underlying geometric attractors.
\end{itemize}

Standard metric or network-based descriptions alone cannot account for these phenomena, motivating exploration of a substrate-level geometric framework.

\section{Core Symbolic Concepts}
VFD models the substrate as a recursive geometric medium supporting stable vibrational identity patterns. The following elements are *symbolic metaphors only*.

\subsection{$\varphi$-Scaled Shells}
Stable identities are represented as nested shells with radii scaled by powers of the golden ratio $\varphi$. These shells are symbolic placeholders for regions where standing-wave recurrence remains stable.

\subsection{Torsional Intervals}
Instead of conventional distances, VFD symbolically uses torsional intervals: angular relationships within a recursive geometric lattice. Synchrony arises when these intervals close under $\varphi$-proportional rotation.

\subsection{Recursive Pattern Stability}
An entity---particle, molecule, neural oscillation, or cognitive structure---is symbolically framed as a recurrence pattern that reappears within $\varphi$-scaled shells in the vibrational substrate.

\section{Symbolic Applications}
The symbolic geometry of VFD offers conceptual interpretations across domains:

\subsection{Neural Synchrony and Cross-Brain Geometry}
Coherence between neural systems may reflect shared geometric invariants rather than purely dynamical coupling. $\varphi$-scaled torsional intervals symbolically correspond to observed synchrony ratios.

\subsection{Molecular Coherence and Vibrational Guidance}
Molecular vibrational interference patterns can be symbolically interpreted as standing-wave behaviour constrained by geometric shells.

\subsection{Emergent Agency}
Agents---cellular, behavioural, or cognitive---may be understood as coherent vibrational identities stabilised by geometric recurrence rather than algorithmic computation.

\section{Relation to Existing Research}
VFD is conceptually compatible with emerging explorations of:

\begin{itemize}
    \item Geometric invariants in interbrain coherence networks.
    \item Wave-based and geometry-driven interpretations of molecular dynamics.
    \item Conceptual frameworks linking geometry to cognition.
    \item Shape-based attractors in neuroscience and morphogenesis.
\end{itemize}

VFD's unique contribution is a unified symbolic geometric substrate linking coherence phenomena across scales.

\section{Closing Remarks}
This Short Symbolic Review introduces only the conceptual layer of VFD and intentionally excludes:
\begin{itemize}
    \item generative mathematical operators,
    \item derivations of physical constants,
    \item substrate dynamics,
    \item computational frameworks.
\end{itemize}
Its role is to provide an accessible entry point for researchers exploring geometry-rooted approaches to coherence and identity.

\vspace{1cm}
\noindent For additional symbolic notes and safe public documents, see the accompanying repository.

\end{document}
